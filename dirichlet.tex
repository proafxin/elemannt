\documentclass[elemannt.tex]{subfile}

\begin{document}
    \section{General Convolution and Dirichlet Hyperbola Method}
    In this chapter, we will discuss Dirichlet convolution and its generalization, use Dirichlet derivative to prove the Selberg identity, establish some results using generalized convolution and finally, prove the fundamental identity of Selberg.
    
    We proved before that
    	\begin{align*}
    		\sum\limits_{n\leq x}\tau(n)
    			& = 2\sum\limits_{n\leq \sqrt{x}}\floor{\dfrac{x}{n}}-\floor{\sqrt{x}}^{2}
    	\end{align*}
    In a similar manner, we can also prove the following.
    	\begin{align*}
    		\sum\limits_{n\leq x}\sigma(n)
    			& = \dfrac{1}{2}\parenthesis{\sum\limits_{n\leq \sqrt{x}}\floor{\dfrac{x}{n}}+\sum\limits_{n\leq \sqrt{x}}(2n+1)\floor{\dfrac{x}{n}}-\floor{\sqrt{x}}^{2}-\floor{\sqrt{x}}^{3}}
    	\end{align*}
    Note that, in both cases, we are able to express the partial sum of a multiplicative function up to $x$ in terms of a combination of some partial sums of some other functions up to $\sqrt{x}$. The generalization of this method is known as the \index{Dirichlet hyperbola method}\textit{Dirichlet hyperbola method}.
    	\begin{theorem}[Dirichlet Hyperbola Method]\label{thm:hyperbola}
    		Let $f$ and $g$ be arithmetic functions. If $h$ is the Dirichlet convolution of $f$ and $g$, then
    			\begin{align*}
    				\sum\limits_{n\leq x}h(n)
    					& = \sum\limits_{n\leq a}f(n)G\parenthesis{\dfrac{x}{n}}+\sum\limits_{n\leq b}g(n)F\parenthesis{\dfrac{x}{n}}-F(a)G(b)
    			\end{align*}
    		where $F$ and $G$ are the partial sums of $f$ and $g$ respectively.
    			\begin{align*}
    				F(x)
    					& = \sum\limits_{n\leq x}f(n)\\
    				G(x)
    					& = \sum\limits_{n\leq x}g(n)
    			\end{align*}
    		Specially when $a=b$,
    			\begin{align*}
    				\sum\limits_{de\leq x}f(d)g(e)
    					& = \sum\limits_{n\leq \sqrt{x}}\parenthesis{f(n)G\parenthesis{\dfrac{x}{n}}+g(n)F\parenthesis{\dfrac{x}{n}}}-F(\sqrt{x})G(\sqrt{x})
    			\end{align*}
    	\end{theorem}
    Next, we will discuss generalizations of Dirichlet convolution. Let $f$ and $g$ be arithmetic functions such that $g(x)=0$ if $0<x<1$. Then the \index{general convolution}\textit{general convolution} of $f$ and $g$ is
    	\begin{align*}
    		f\circ g(x)
    			& = \sum\limits_{n\leq x}f(n)g\parenthesis{\dfrac{x}{n}}
    	\end{align*}
    We can easily prove the following.
    	\begin{theorem}[General convolution theorem]\label{thm:genconv}
    		Let $f,g$ and $h$ be arithmetic functions. Then
    			\begin{align*}
    				(f\ast g)\circ h
    					& = f\circ(g\circ h)
    			\end{align*}
    	\end{theorem}
    From this, we can also get the general M\"{o}bius inversion formula.
    	\begin{theorem}
    		Let $f,g$ be an arithmetic functions and $f^{-1}$ be the Dirichlet inverse of $f$. If
    			\begin{align*}
    				G(x)
    					& = \sum\limits_{n\leq x}f(n)g\parenthesis{\dfrac{x}{n}}
    			\end{align*}
    		then
    			\begin{align*}
    				g(x)
    					& = \sum\limits_{n\leq x}f^{-1}(n)G\parenthesis{\dfrac{x}{n}}
    			\end{align*}
    	\end{theorem}
\end{document}