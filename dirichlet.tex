\documentclass[elemannt.tex]{subfile}

\begin{document}
    \section{General Convolution and Dirichlet Hyperbola Method}
    In this chapter, we will first discuss Dirichlet convolution and a generalization. Then we will discuss a variation and another generalization both of which are very useful in many cases.

    We proved before that
    	\begin{align*}
    		\sum\limits_{n\leq x}\tau(n)
    			& = 2\sum\limits_{n\leq \sqrt{x}}\floor{\dfrac{x}{n}}-\floor{\sqrt{x}}^{2}
    	\end{align*}
    In a similar manner, we can also prove the following.
    	\begin{align*}
    		\sum\limits_{n\leq x}\sigma(n)
    			& = \dfrac{1}{2}\parenthesis{\sum\limits_{n\leq \sqrt{x}}\floor{\dfrac{x}{n}}+\sum\limits_{n\leq \sqrt{x}}(2n+1)\floor{\dfrac{x}{n}}-\floor{\sqrt{x}}^{2}-\floor{\sqrt{x}}^{3}}
    	\end{align*}
    Note that, in both cases, we are able to express the partial sum of a multiplicative function up to $x$ in terms of a combination of some partial sums of some other functions up to $\sqrt{x}$. The generalization of this method is known as the \index{Dirichlet hyperbola method}\textit{Dirichlet hyperbola method}.
    	\begin{theorem}[Dirichlet Hyperbola Method]\label{thm:hyperbola}
    		Let $f$ and $g$ be arithmetic functions. If $h$ is the Dirichlet convolution of $f$ and $g$, then
    			\begin{align*}
    				\sum\limits_{n\leq x}h(n)
    					& = \sum\limits_{n\leq a}f(n)G\parenthesis{\dfrac{x}{n}}+\sum\limits_{n\leq b}g(n)F\parenthesis{\dfrac{x}{n}}-F(a)G(b)
    			\end{align*}
    		where $F$ and $G$ are the partial sums of $f$ and $g$ respectively.
    			\begin{align*}
    				F(x)
    					& = \sum\limits_{n\leq x}f(n)\\
    				G(x)
    					& = \sum\limits_{n\leq x}g(n)
    			\end{align*}
    		Specially when $a=b$,
    			\begin{align*}
    				\sum\limits_{de\leq x}f(d)g(e)
    					& = \sum\limits_{n\leq \sqrt{x}}\parenthesis{f(n)G\parenthesis{\dfrac{x}{n}}+g(n)F\parenthesis{\dfrac{x}{n}}}-F(\sqrt{x})G(\sqrt{x})
    			\end{align*}
    	\end{theorem}
    Next, we will discuss generalizations of Dirichlet convolution. Let $f$ and $g$ be arithmetic functions such that $g(x)=0$ if $0<x<1$. Then the \index{general convolution}\textit{general convolution} of $f$ and $g$ is
    	\begin{align*}
    		f\circ g(x)
    			& = \sum\limits_{n\leq x}f(n)g\parenthesis{\dfrac{x}{n}}
    	\end{align*}
    We can easily prove the following.
    	\begin{theorem}[General convolution theorem]\label{thm:genconv}
    		Let $f,g$ and $h$ be arithmetic functions. Then
    			\begin{align*}
    				(f\ast g)\circ h
    					& = f\circ(g\circ h)
    			\end{align*}
    	\end{theorem}
    From this, we can also get the general M\"{o}bius inversion formula.
    	\begin{theorem}
    		Let $f,g$ be an arithmetic functions and $f^{-1}$ be the Dirichlet inverse of $f$. If
    			\begin{align*}
    				G(x)
    					& = \sum\limits_{n\leq x}f(n)g\parenthesis{\dfrac{x}{n}}
    			\end{align*}
    		then
    			\begin{align*}
    				g(x)
    					& = \sum\limits_{n\leq x}f^{-1}(n)G\parenthesis{\dfrac{x}{n}}
    			\end{align*}
    	\end{theorem}
    \section{A Variation of Generalized Convolution}
    We will now consider a slight variation of generalized convolution.
    \begin{align*}
    	(f\diamond g)(x)
    	& = \sum_{n\leq x}f(n)g\parenthesis{\floor{\dfrac{x}{n}}}
    \end{align*}

    \begin{theorem}\label{thm:genconvhyp}
    	Let $f$ and $g$ be arithmetic functions. Then
    	\begin{align*}
    		(f\diamond g)(x)
    		& = \sum_{n\leq\sqrt{x}}g(n)\parenthesis{F\parenthesis{\floor{\dfrac{x}{n}}}-F\parenthesis{\floor{\dfrac{x}{n+1}}}}+\sum_{n\leq x/(\floor{\sqrt{x}+1})}f(n)g\parenthesis{\floor{\dfrac{x}{n}}}
    	\end{align*}
    	We can also write it as
    	\begin{align*}
    		(f\diamond g)(x)
    		& = \sum_{n\leq\sqrt{x}}g(n)\parenthesis{F\parenthesis{\floor{\dfrac{x}{n}}}-F\parenthesis{\floor{\dfrac{x}{n+1}}}}+f(n)g\parenthesis{\floor{\dfrac{x}{n}}}-\mathfrak{B}(x)
    	\end{align*}
    	where
    	\begin{align*}
    		\mathfrak{B}(x)
    		& =
    		\begin{cases}
    			f(\floor{\sqrt{x}})g\parenthesis{\floor{\dfrac{x}{\floor{\sqrt{x}}}}}& \mbox{ if }x< \floor{\sqrt{x}}(\floor{\sqrt{x}}+1)\\
    			0& \mbox{ otherwise}
    		\end{cases}
    	\end{align*}
    \end{theorem}

    \begin{proof}
    	Consider the integers
    	\begin{align*}
    		\floor{\dfrac{x}{1}},\ldots,\floor{\dfrac{x}{\floor{x}}}
    	\end{align*}
    	All $n\leq x$ does not appear in this list, only $n\leq\sqrt{x}$ and numbers of the form $\floor{x/n}$ for $n\leq\sqrt{x}$ appear on this list. In fact, at most $2\floor{\sqrt{n}}$ distinct values appear in this list.
    \end{proof}
    For the rest of this section, consider $\diamond$ for arbitrary $f$ and $g$,
    \begin{align*}
    	F(x)
    	& = \sum_{n\leq x}f(n)\\
    	& = O(x^{\xi})\\
    	g(x)
    	& = \floor{x}^{k}
    \end{align*}
    for a constant $\xi$ and a fixed positive integer. Then we have
    \begin{align*}
    	(f\diamond g)(x)
    	& = \sum_{n\leq x}f(n)\floor{\dfrac{x}{n}}^{k}\\
    	& = \sum_{n\leq \sqrt{x}}n^{k}\parenthesis{F\parenthesis{\floor{\dfrac{x}{n}}}-F\parenthesis{\floor{\dfrac{x}{n+1}}}}+f(n)\floor{\dfrac{x}{n}}^{k}-\mathfrak{B}(x)\\
    	& = \sum_{n\leq \sqrt{x}}n^{k}\bigo{\floor{\dfrac{x}{n}}^{\xi}}+f(n)\floor{\dfrac{x}{n}}^{k}-\mathfrak{B}(x)\\
    	& = x^{\xi}\bigo{\sum_{n\leq \sqrt{x}}\dfrac{n^{k}}{n^{\xi}}}+\sum_{n\leq \sqrt{x}}f(n)\parenthesis{\parenthesis{\dfrac{x}{n}}+O(1)}^{k}-\mathfrak{B}(x)\\
    	& = x^{\xi}\bigo{\sum_{n\leq \sqrt{x}}\dfrac{n^{k}}{n^{\xi}}}+\sum_{n\leq \sqrt{x}}f(n)\parenthesis{\parenthesis{\dfrac{x}{n}}^{k}+\bigo{\parenthesis{\dfrac{x}{n}}^{k-1}}}-\mathfrak{B}(x)\\
    	& = x^{\xi}\bigo{\sum_{n\leq \sqrt{x}}\dfrac{n^{k}}{n^{\xi}}}+x^{k}\sum_{n\leq \sqrt{x}}\dfrac{f(n)}{n^{k}}+\bigo{x^{k-1}\sum_{n\leq \sqrt{x}}\dfrac{f(n)}{n^{k-1}}}-\mathfrak{B}(x)
    \end{align*}
    Now we need to focus on the following two sums.
    \begin{align*}
    	\mathfrak{M}_{s}(x)
    	& = \sum_{n\leq x}\dfrac{n^{s}}{n^{\xi}}\\
    	\mathfrak{G}_{s}(x)
    	& = \sum_{n\leq x}\dfrac{f(n)}{n^{s}}
    \end{align*}
    where $s\geq1$. Then
    \begin{align}
    	(f\diamond g)(x)
    	& = x^{\xi}\bigo{\mathfrak{M}_{k}(\sqrt{x})}+x^{k}\mathfrak{G}_{k}(\sqrt{x})+\bigo{x^{k-1}\mathfrak{G}_{k-1}(\sqrt{x})}-\mathfrak{B}(x)\label{eqn:diamond}
    \end{align}
	We can use \autoref{thm:zetapositive} for computing $\mathfrak{M}$. If $s+1<\xi$,
    \begin{align}
    	\mathfrak{M}_{s}(x)
    	& = \sum_{n\leq x}\dfrac{1}{n^{\xi-s}}\nonumber\\
    	& = \dfrac{x^{1+s-\xi}}{1+s-\xi}+\zeta(\xi-s)+\bigo{x^{s-\xi}}\label{eqn:Us}
    \end{align}
    If $s+1=\xi$, from \autoref{thm:harmonicsum},
    \begin{align*}
    	\mathfrak{M}_{s}(x)
    	& = \log{x}+C+\bigo{\dfrac{1}{x}}
    \end{align*}
    The case $\xi-1<s<\xi$ is possible if and only if $\xi$ is not an integer and $s=\floor{\xi}$ which can be taken care of in the same manner as \eqref{eqn:Us}.  We can now assume $s\geq\xi$. In this case, $s-\xi\geq0$ and from \autoref{thm:lehmer}
    \begin{align*}
    	\mathfrak{M}_{s}(x)
    	& = \sum_{n\leq x}n^{s-\xi}\\
    	& = \dfrac{x^{s-\xi+1}}{s-\xi+1}+\bigo{x^{s-\xi}}
    \end{align*}
    For handling $\mathfrak{G}$, we will use \autoref{thm:abel}.
    \begin{align*}
    	\mathfrak{G}_{s}(x)
    	& = \dfrac{F(x)}{x^{s}}+s\int_{1}^{x}F(t)t^{-s-1}dt\\
    	& = \bigo{x^{\xi-s}}+s\bigo{\int_{1}^{x}t^{\xi-s-1}dt}
    \end{align*}
    Thus, we have
    \begin{align*}
    	\mathfrak{M}_{s}(x),\mathfrak{G}_{s}(x)
    	& =
    	\begin{cases}
    		\log{x}+C+\bigo{\dfrac{1}{x}},\bigo{x}& \mbox{ if }s+1=\xi\\
    		\dfrac{x^{1+s-\xi}}{1+s-\xi}+\zeta(\xi-s)+\bigo{x^{s-\xi}},\bigo{x^{\xi-s}}& \mbox{ if }s<\xi\mbox{ and }s+1\neq\xi\\
    		x+\bigo{1},\bigo{\log{x}}& \mbox{ if }s=\xi\\
    		\dfrac{x^{s-\xi+1}}{s-\xi+1}+\bigo{x^{s-\xi}}, \bigo{x^{\xi-s}}& \mbox{ if }s>\xi
    	\end{cases}
    \end{align*}
    Plugging these back in \eqref{eqn:diamond}, we get the following result.
    \begin{theorem}\label{thm:genconvvar}
    	Let $f$ and $g$ be arithmetic functions such that $F(x) = \sum_{n\leq x}f(n)= \bigo{x^{\xi}}, g(x)= \floor{x}^{k}$. Then
    	\begin{align*}
    		(f\diamond g)(x)
	    		& =
	    		\begin{cases}
	    			\bigo{x^{k+1}\log{x}}& \mbox{ if }k+1=\xi\\
	    			\bigo{x^{\frac{k+\xi+1}{2}}+x^{\xi}\zeta(\xi-k)}&\mbox{ if }k<\xi,k+1\neq\xi\\
	    			\bigo{x^{k+\frac{1}{2}}}&\mbox{ if }k=\xi\\
	    			\bigo{x^{\frac{k+\xi+1}{2}}}&\mbox{ if }k>\xi
	    		\end{cases}
    	\end{align*}
    \end{theorem}
\end{document}