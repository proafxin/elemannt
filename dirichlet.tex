\documentclass[elemannt.tex]{subfile}

\begin{document}
    \chapter{Dirichlet Convolution and Generalization}
    In this chapter, we will discuss Dirichlet convolution and its generalization, use Dirichlet derivative to prove the Selberg identity, establish some results using generalized convolution and finally, prove the fundamental identity of Selberg.
        \begin{definition}[Dirichlet product]
            For two arithmetic functions $f$ and $g$, the \textit{Dirichlet product} or \index{Dirichlet convolution}\textit{Dirichlet convolution} of $f$ and $g$ is defined as
                \begin{align*}
                    f\ast g
                        & = \sum_{d\mid n}f(d)g\left(\dfrac{n}{d}\right)
                \end{align*}
        \end{definition}
    It is not so easy to see how the idea of Dirichlet convolution originates even though it is highly used in number theory. We can connect its origin with the \index{Zeta function}zeta function.
        \begin{align*}
            \zeta(s)
                & = \dfrac{1}{1^{s}}+\dfrac{1}{2^{s}}+\ldots\\
                & = \sum_{i\geq 1}\dfrac{1}{i^{s}}
        \end{align*}
    
\end{document}