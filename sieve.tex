\documentclass[elemannt.tex]{subfile}

\begin{document}
	\chapter[Sieve Theory]{A Modest Introduction to Sieve Theory}
	A composite positive integer $n$ has at least one prime factor not exceeding $\sqrt{x}$. Thus, the number of primes in the interval $[\sqrt{x},x]$ is
		\begin{align*}
			\pi(x)-\pi(\sqrt{x})+1
				& = \floor{x}-\sum_{p\leq x}\floor{\dfrac{x}{p}}+\sum_{p_{1}<p_{2}\leq x}\floor{\dfrac{x}{p_{1}p_{2}}}-\sum_{p_{1}<p_{2}<p_{3}\leq x}\floor{\dfrac{x}{p_{1}p_{2}p_{3}}}+\ldots\\
				& = \sum_{n\leq x}\mu(n)\floor{\dfrac{x}{n}}
		\end{align*}
	Now, $\floor{x}=x+O(1)$, so
		\begin{align*}
			\pi(x)-\pi(\sqrt{x})+1
				& = x\sum_{\substack{n\leq x\\\rho(n)\leq \sqrt{x}}}\dfrac{\mu(n)}{n}+O\parenthesis{\sum_{\substack{n\leq x\\\rho(n)\leq\sqrt{x}}}\mu(n)}\\
				& = x\prod_{p\leq\sqrt{x}}\parenthesis{1-\dfrac{1}{p}}+O(2^{\pi(\sqrt{x})})
		\end{align*}
	The last line is true since there are $\pi(\sqrt{x})$ primes not exceeding $\sqrt{x}$ and $|\mu(n)|=1$ for all square-free $n\leq x$ such that $\rho(n)\leq\sqrt{x}$. However, this is not particularly useful so we want to improve on this. This is the starting point for sieves. Let us generalize this concept first. Consider a set of integers $A$. $\mathfrak{M},\mathfrak{G}$
	\section{Brun's Sieve}
	\textcite{brun_1915} also see \textcite{brun_1920}
	\section{Selberg's Sieve}
	\section{Tur\'{a}n's Method}
\end{document}