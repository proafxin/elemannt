\documentclass[elemannt.tex]{subfile}

\begin{document}
	\chapter{Bertrand to Tchebycheff}
	We said before that \textit{almost} all natural numbers are composite. A major objective of this book is to discuss how often the primes occur. The same question has bugged mathematicians for a centuries. It was Gauss who first observed that the change in the distribution of primes in every interval $[x,x+1000]$ was around $1/\log{x}$. Thus, the rough estimate
		\begin{align*}
			\pi(x)
				& = \int\limits_{2}^{x}\dfrac{1}{\log{t}}dt
		\end{align*}
	was made which is now known as \index{Logarithmic integral}\textit{logarithmic integral}. Gauss conjectured (see \textcite[Page 37]{landau_1911}) around 1792 or 1793 that
		\begin{align*}
			\lim\limits_{x\to\infty}\dfrac{\pi(x)}{\dfrac{x}{\log{x}}}
				& = 1
		\end{align*}
	\textcite{tchebycheff_1852} is the first one to make any substantial progress on the matter.
		\begin{definition}[Tchebycheff's theta function]
			Tchebycheff function of the first kind or \index{Tchebycheff's theta function}\textit{Tchebycheff's theta function} is defined as
				\begin{align*}
					\vartheta(x)
						& = \sum_{p\leq x}\log{p}
				\end{align*}
		\end{definition}

		\begin{definition}[Tchebycheff's psi function]
			Tchebycheff function of the second kind or \index{Tchebycheff's psi function}\textit{Tchebycheff's psi function} is defined as
				\begin{align*}
					\psi(x)
						& = \sum_{n\leq x}\Lambda(n)\\
						& = \sum_{p^{e}\leq x}\log{p}\\
						& = \sum_{p\leq x}\left\lfloor{\dfrac{\log{x}}{\log{p}}}\right\rfloor \log{p}
				\end{align*}
		\end{definition}
\end{document}