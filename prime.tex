\documentclass[elemannt.tex]{subfile}

\begin{document}
	\textcite{euler_1737} proved this first although his method was a bit questionable.
		\begin{theorem}[Divergence of sum of reciprocals of primes]
			The sum
				\begin{align*}
					\sum_{p}\dfrac{1}{p}
						& = \dfrac{1}{2}+\dfrac{1}{3}+\dfrac{1}{5}+\ldots
				\end{align*}
			where the sum is taken over all primes diverges.
		\end{theorem}
	This proof is again inspired by \textcite[Theorem 114]{landau_1969}.
		\begin{proof}
			From \nameref{thm:harmonicsum}, we already know that
				\begin{align*}
					\sum_{n\geq 1}\dfrac{1}{n}
						& = \prod_{p}\dfrac{1}{1-\dfrac{1}{p}}
				\end{align*}
			and that this sum diverges. Now,
				\begin{align*}
					\log\parenthesis{\sum_{n\geq 1}\dfrac{1}{n}}
						& = \sum_{p}\parenthesis{-\log\parenthesis{1-\dfrac{1}{p}}}
				\end{align*}
			Setting $\eta:=1/p$ and using
				\begin{align*}
					\log\parenthesis{1-x}
						& = -x-\dfrac{x^{2}}{2}-\dfrac{x^{3}}{3}-\ldots
				\end{align*}
			we have
				\begin{align*}
					\log\parenthesis{\sum_{n\geq 1}\dfrac{1}{n}}
						& = \sum_{p}\parenthesis{\eta+\dfrac{\eta^{2}}{2}+\dfrac{\eta^{3}}{3}+\ldots}\\
						& < \sum_{p}\parenthesis{\eta+\eta^{2}+\ldots}\\
						& = \sum_{p}\dfrac{\eta}{1-\eta}\\
						& < 2\sum_{p}\eta\\
						& = 2\sum_{p}\dfrac{1}{p}
				\end{align*}
			If $\sum_{p}\frac{1}{p}$ does not diverge, then
				\begin{align*}
					\log\parenthesis{\sum_{n\geq 1}\dfrac{1}{n}}
						& < C
				\end{align*}
			for a constant $C$. Thus,
				\begin{align*}
					\sum_{n\geq 1}\dfrac{1}{n}
						& <e^{C}
				\end{align*}
			and does not converge either. This is impossible. So, the original sum must diverge.
		\end{proof}
	\textcite{mertens_1874} actually proved that
		\begin{align*}
			\sum_{p\leq x}\dfrac{1}{p}
				& = \log{\log{x}}
		\end{align*}
	This is the first formal proof since Euler's method wasn't exactly clean. \textcite[Page 228]{euler_1748} uses
		\begin{align*}
			\log\parenthesis{\dfrac{1}{1-x}}
				& = \sum_{n\geq 1}\dfrac{x^{n}}{n}
		\end{align*}
	and sets $x:=1$ to conclude
		\begin{align*}
			\sum_{n\geq 1}\dfrac{1}{n}
				& = \infty
		\end{align*}
	This is indeed true, however, Euler's statement is vague. Today it is known that
		\begin{align*}
			\sum_{p\leq x}\dfrac{1}{p}
				& \geq \log{\log\parenthesis{x+1}}-\dfrac{\pi^{2}}{6}
		\end{align*}
	More precise results and their rigorous proofs seem to be given by \textcite{mertens_1874}.
		\begin{theorem}[Mertens' theorems]\label{thm:mertens}
			Let $x$ be a positive real number. As $x\to\infty$,
				\begin{align}
					\sum_{p\leq x}\dfrac{1}{p}
						& = \log{\log{x}}+B+\bigo{\dfrac{1}{x}}\label{eqn:mertens1}\\
					\sum_{p\leq x}\dfrac{\log{p}}{p}
						& = \log{x}+O(1)\label{eqn:mertens2}\\
					\prod_{p\leq x}\left(1-\dfrac{1}{p}\right)
						& \sim \dfrac{e^{-\gamma}}{\log{x}}\label{eqn:mertens3}
				\end{align}
			where $B$ is a constant and $\gamma$ is Euler-Mascheroni constant.
		\end{theorem}

		\begin{proof}

		\end{proof}
\end{document}