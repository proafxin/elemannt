\documentclass[elemannt.tex]{subfile}

\begin{document}
	\textcite{euler_1737} proved this first although his method was a bit questionable.
		\begin{theorem}[Divergence of sum of reciprocals of primes]\label{thm:primereciprocal}
			The sum
				\begin{align*}
					\sum_{p}\dfrac{1}{p}
						& = \dfrac{1}{2}+\dfrac{1}{3}+\dfrac{1}{5}+\ldots
				\end{align*}
			where the sum is taken over all primes diverges.
		\end{theorem}
	This proof is inspired by \textcite[Theorem 114]{landau_1969}.
		\begin{proof}
			From \nameref{thm:harmonicsum}, we already know that
				\begin{align*}
					\sum_{n\geq 1}\dfrac{1}{n}
						& = \prod_{p}\dfrac{1}{1-\dfrac{1}{p}}
				\end{align*}
			and that this sum diverges. Now,
				\begin{align*}
					\log\parenthesis{\sum_{n\geq 1}\dfrac{1}{n}}
						& = \sum_{p}\parenthesis{-\log\parenthesis{1-\dfrac{1}{p}}}
				\end{align*}
			Setting $\eta:=1/p$ and using
				\begin{align*}
					\log\parenthesis{1-x}
						& = -x-\dfrac{x^{2}}{2}-\dfrac{x^{3}}{3}-\ldots
				\end{align*}
			we have
				\begin{align*}
					\log\parenthesis{\sum_{n\geq 1}\dfrac{1}{n}}
						& = \sum_{p}\parenthesis{\eta+\dfrac{\eta^{2}}{2}+\dfrac{\eta^{3}}{3}+\ldots}\\
						& < \sum_{p}\parenthesis{\eta+\eta^{2}+\ldots}\\
						& = \sum_{p}\dfrac{\eta}{1-\eta}\\
						& < 2\sum_{p}\eta\\
						& = 2\sum_{p}\dfrac{1}{p}
				\end{align*}
			If $\sum_{p}\frac{1}{p}$ does not diverge, then
				\begin{align*}
					\log\parenthesis{\sum_{n\geq 1}\dfrac{1}{n}}
						& < C
				\end{align*}
			for a constant $C$. Thus,
				\begin{align*}
					\sum_{n\geq 1}\dfrac{1}{n}
						& <e^{C}
				\end{align*}
			and does not converge either. This is impossible. So, the original sum must diverge.
		\end{proof}
	\textcite{mertens_1874} actually proved that
		\begin{align*}
			\sum_{p\leq x}\dfrac{1}{p}
				& = \log{\log{x}}
		\end{align*}
	This is the first formal proof since Euler's method wasn't exactly clean. \textcite[Page 228]{euler_1748} uses
		\begin{align*}
			\log\parenthesis{\dfrac{1}{1-x}}
				& = \sum_{n\geq 1}\dfrac{x^{n}}{n}
		\end{align*}
	and sets $x:=1$ to conclude
		\begin{align*}
			\sum_{n\geq 1}\dfrac{1}{n}
				& = \infty
		\end{align*}
	This is indeed true, however, Euler's statement is vague. So this is not usually considered a rigorous proof of this result.
		\begin{theorem}
			If $x\to\infty$, then $\pi(x)=\bigo{\frac{x}{\log{\log{x}}}}$. A weaker statement is $\pi(x)=o(x)$.
		\end{theorem}

		\begin{proof}
			Let $\xi$ be a real number such that $2<\xi<x$ and the primes not exceeding $\xi$ are $p_{1},\ldots,p_{k}$. The number of positive integers not divisible by any of $p_{1},\ldots,p_{k}$ in the interval $[\xi+1,\ldots,x]$ is
				\begin{align*}
					\varphi(x,\xi)
						& = x-\sum\floor{\dfrac{x}{p_{i}}}+\sum_{i<j}\floor{\dfrac{x}{p_{i}p_{j}}}-\ldots
				\end{align*}
			If $\xi\geq\sqrt{x}$, then we have $\varphi(x,\xi)=\pi(x)-k+1$. In general, $\pi(x)-r+1\leq \varphi(x,\xi)$ holds. Now,
				\begin{align*}
					\varphi(x,\xi)
						& \leq x-\sum\left(\dfrac{x}{p_{i}}\right)+1+\sum_{i<j}\left(\dfrac{x}{p_{i}p_{j}}\right)+1-\ldots\\
						& = x\prod_{i=1}^{k}\left(1-\dfrac{1}{p_{i}}\right)+2^{k}\\
						& < x\prod_{p\leq\xi}\left(1-\dfrac{1}{p}\right)+2^{\xi}
				\end{align*}
			Note that the choice of $\xi$ is arbitrary and we can easily choose $2^{\xi}+r-1=o(x)$ or $\xi=c\log{x}$ for some constant $c$. Also,
				\begin{align*}
					\prod_{p\leq x}\left(1-\dfrac{1}{p}\right)^{-1}
						& = \prod_{p\leq x}\left(1+\dfrac{1}{p}+\dfrac{1}{p^{2}}+\ldots\right)\\
						& > \sum_{n\leq x}\dfrac{1}{n}\\
						& > \log{x}
				\end{align*}
			so
				\begin{align*}
					\prod_{p\leq \xi}\left(1-\dfrac{1}{p}\right)
						& < \dfrac{1}{\log{c\log{x}}}\\
						& = \dfrac{1}{\log{c}+\log{\log{x}}}\\
						& < \dfrac{1}{\log{\log{x}}}
				\end{align*}
			Therefore,
				\begin{align*}
					\pi(x)
						& \leq \varphi(x,\xi)+r-1\\
						& <\frac{x}{\log{\log{x}}}+o(x)\\
					\pi(x)
						& = \bigo{\dfrac{x}{\log{\log{x}}}}
				\end{align*}
			and evidently, $\pi(x)=o(x)$ since
				\begin{align*}
					\lim_{x\to\infty}\dfrac{\dfrac{x}{\log{\log{x}}}}{x}
						& = 0
				\end{align*}
		\end{proof}

\end{document}