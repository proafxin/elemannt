\documentclass[elemannt.tex]{subfile}

\begin{document}
	\chapter{Arithmetic Functions}
	In this chapter, we will discuss some generalized arithmetic functions and their asymptotic behavior. We will skip discussing the basic definitions since they are common in most introductory number theory texts. $f:\mathbb{N}\to\mathbb{N}$.
		\begin{definition}[Summatory function]
			For an arithmetic function $f$, the \textit{summatory function} of $f$ is defined as
				\begin{align*}
					F(n)
						& = \sum_{d\mid n}f(d)
				\end{align*}
		\end{definition}
	Note that the number of divisor function $\tau(n)$ is the summatory function of the unit function $u(n)=1$. The sum of divisor function $\sigma(n)$ is the summatory function of the invariant function $f(n)=n$. An interesting property that we will repeatedly use is that
		\begin{align*}
			\sum_{i=1}^{n}F(i)
				& = \sum_{i=1}^{n}\sum_{d\mid i}f(d)\\
				& = \sum_{i=1}^{n}\left\lfloor{\dfrac{n}{i}}\right\rfloor f(i)
		\end{align*}
	Here, the last equation is true because there are $\lfloor{n/i}\rfloor$ multiples of $i$ not exceeding $n$. We can 
		\begin{definition}[Generalized sum of divisor]
			The \index{Generalized sum of divisor}\textit{generalized sum of divisor} function can be defined as
				\begin{align*}
					\sigma_{k}(n)
					& = \sum_{d\mid n}d^{k}
			\end{align*}
		\end{definition}
	We can write $\tau(n)=\sum_{ab=n}1$.
		\begin{definition}[Generalized number of divisor]
			The \index{Generalized number of divisor}\textit{generalized number of divisor} function is defined as
				\begin{align*}
					\tau_{k}(n)
						& = 
				\end{align*}
		\end{definition}
	We will focus on the following function for now.
		\begin{align*}
			\sum_{n\leq x}\tau_{k}(n)
		\end{align*}
\end{document}