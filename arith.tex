\documentclass[elemannt.tex]{subfile}

\begin{document}
	\chapter{Arithmetic Functions}
	In this chapter, we will discuss some generalized arithmetic functions and their asymptotic behavior. We will skip discussing the basic definitions since they are common in most introductory number theory texts. But for the sake of completeness, check \gls{mobius}, \gls{erdos} in the Glossary.
		\begin{definition}[Summatory function]
			For an arithmetic function $f$, the \textit{summatory function} of $f$ is defined as
				\begin{align*}
					F(n)
						& = \sum_{d\mid n}f(d)
				\end{align*}
		\end{definition}
	Note that the number of divisor function $\tau(n)$ is the summatory function of the unit function $u(n)=1$. The sum of divisor function $\sigma(n)$ is the summatory function of the invariant function $f(n)=n$. An interesting property that we will repeatedly use is that
		\begin{align*}
			\sum_{i=1}^{n}F(i)
				& = \sum_{i=1}^{n}\sum_{d\mid i}f(d)\\
				& = \sum_{i=1}^{n}\left\lfloor{\dfrac{n}{i}}\right\rfloor f(i)
		\end{align*}
	Here, the last equation is true because there are $\lfloor{n/i}\rfloor$ multiples of $i$ not exceeding $n$. We can 
		\begin{definition}[Generalized sum of divisor]
			The \index{Generalized sum of divisor}\textit{generalized sum of divisor} function can be defined as
				\begin{align*}
					\sigma_{k}(n)
					& = \sum_{d\mid n}d^{k}
			\end{align*}
		\end{definition}
	Recall that the number of divisor function $\tau(n)=\sum_{ab=n}1$. We can generalize this as follows.
		\begin{definition}[Generalized number of divisor]
			The \index{Generalized number of divisor}\textit{generalized number of divisor} function is defined as
				\begin{align*}
					\tau_{k}(n)
						& = \sum_{d_{1}\cdots d_{k}=n}1
				\end{align*}
			So $\tau_{k}(n)$ is the number of ways to write $n$ as a product of $k$ positive integers. 
		\end{definition}
	At this point, we will define some asymptotic notions.
		\begin{definition}[Big O]
			Let $f$ and $g$ be two real or complex valued functions. We say that
				\begin{align*}
					f(x)
					& = O(g(x))
				\end{align*}
			if there is a constant $C$ such that
				\begin{align*}
					|f(x)|
					& \leq Cg(x)
				\end{align*}
			for all sufficiently large $x$. It is also written as $f(x)\ll g(x)$  or $g(x)\gg f(x)$.
		\end{definition}
	Some trivial examples are $x^{2}=O(x^{3})$, $x+1=O(x)$ and $x^{2}+x=O(x^{2})$. We usually want $g(x)$ to be as small as possible to avoid triviality.
		\begin{definition}[Small O]
			Let $f$ and $g$ be two real or complex valued functions. Then the following two statements are equivalent
				\begin{align*}
					f(x)
						& = o(g(x))\\
					\iff \lim\limits_{x\to\infty}\dfrac{f(x)}{g(x)}
						& = 0
				\end{align*}
		\end{definition}
	Some trivial examples are $1/x=o(1)$, $x=o(x^{2})$ and $2x^{2}\neq o(x^{2})$.
		\begin{definition}[Equivalence]
			Let $f$ and $g$ be two real or complex valued functions. We say that they are \textbf{asymptotically equivalent} if
				\begin{align*}
					\lim\limits_{x\to\infty}\dfrac{f(x)}{g(x)}
						& = 1
				\end{align*}
			and we denote it by $f\sim g$.
		\end{definition}
	An example is $x^{2}\sim x^{2}+x$. Note the following.
		\begin{align*}
			f
				& \sim g\\
			\iff |f(x)-g(x)|
				& = o(g(x))
		\end{align*}
	We will use these symbols extensively throughout the book. It is advised that the reader properly familiarize themselves with these notations. These symbols make our lives a lot easier very often. 
\end{document}