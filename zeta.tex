\documentclass[elemannt.tex]{subfile}

\begin{document}
	\section{Dirichlet Series and Dirichlet Convolution}\label{sec:zeta}
	We encountered $\zeta$ when we tried to develop an asymptotic for $S_{k}(x)$. The function $\zeta$ has quite a rich history. Today $\zeta$ is mostly called Riemann's zeta function, however, Euler is the first one to investigate this function. Euler started working on $\zeta$ around $1730$. During that period, the value of $\zeta(2)$ was unknown and of high interest among prominent mathematicians. \textcite{ayoub_1974} is a very good read on this subject. Euler's first contribution in this matter is \textcite{euler_1738} where he proves that $\zeta(2)\approx 1.644934$. The paper was first presented to the St. Petersburg Academy on March 5, $1731$ and republished in \textcite{euler_2020_a}. \textcite{euler_1744} (republished in \textcite{euler_2020_b}) proves the following fundamental result which essentially gives a new proof of infinitude of primes.
		\begin{theorem}[Euler's identity]
			Let $s$ be a positive integer. Then
				\begin{align*}
					\zeta(s)
						& = \prod_{p}\dfrac{p^{s}}{p^{s}-1}
				\end{align*}
			where $p$ extends over all primes.
		\end{theorem}
	One of the results in \textcite{euler_1744} is the following which we shall prove later.
		\begin{align*}
			\sum_{n\leq x}\dfrac{1}{p}
			& \sim \log{\sum_{n\leq x}\dfrac{1}{x}}
		\end{align*}
	Here, $\sim$ is the asymptotic equivalence we have already defined. Even though Euler is the main architect behind the development of $\zeta$, \textcite{riemann_1859} first considered $\zeta$ for complex $s$ instead of real $s$ only. By tradition, we write $s=\sigma+it$ where $\sigma=\Re(s)$ is the real part of $s$ and $t=\Im(s)$ is the imaginary part of $s$.
		\begin{definition}[Dirichlet series]
			For a complex number $s$, a \index{Dirichlet series}\textit{Dirichlet series} is a series of the form
				\begin{align*}
					\mathfrak{D}_{a}(s)
						& = \sum_{n\geq 1}\dfrac{a(n)}{n^{s}}
				\end{align*}
		\end{definition}
	So, $\zeta$ is a special case of $\mathfrak{D}$ when $a(n)=1$ for all $n$. \textcite[$\S1$, Page $1$]{hardy_riesz_1915} considers the following as \index{General Dirichlet series}\textit{general Dirichlet series}
		\begin{align}
			\sum_{n\geq 1}a_{n}e^{-\lambda_{n}s}\label{eqn:gendir}
		\end{align}
	where $(\lambda_{n})$ is an increasing sequence of real numbers. Following this, \textcite{hardy_riesz_1915} calls $\mathfrak{D}$ the \textit{ordinary Dirichlet series} when $\lambda_{n}=\log{n}$. \textcite{lejeune_1879} considers real values of $s$ and proves a number of important theorems. As Hardy states, \textcite{jensen_1884,jensen_1888} discusses the first theorems where $s$ is complex involving the nature of convergence of \ref{eqn:gendir}. \textcite{cahen_1894} makes the \textit{first attempt to construct a systematic theory of the function $\mathfrak{D}_{f}(s)$ although much of the analysis which it contains is open to serious criticism, has served---and possibly just for that reason---as the starting point of most of the later researches in the subject.}
		\begin{definition}[Convergence]
			
		\end{definition}
	\textcite{jensen_1884} proves the following theorem of fundamental importance.
		\begin{theorem}
			Suppose $\mathfrak{D}_{f}(s)$ is convergent for the complex number $s=\omega+it$, then it is convergent for any value of $s$ for which $\Re(s)>\omega$.
		\end{theorem}
	Consider the Dirichlet series for two arithmetic functions $f$ and $g$.
		\begin{align*}
			\mathfrak{D}_{f}(s)
				& = \sum_{n\geq 1}\dfrac{f(n)}{n^{s}}\\
			\mathfrak{D}_{g}(s)
				& = \sum_{n\geq 1}\dfrac{g(n)}{n^{s}}
		\end{align*}
	Then we have
		\begin{align*}
			\mathfrak{D}_{f}(s)\mathfrak{D}_{g}(s)
				& = \sum_{i\geq 1}\dfrac{f(n)}{n^{s}}\sum_{i\geq 1}\dfrac{g(n)}{n^{s}}
		\end{align*}
	Now, imagine we want to write this product as another Dirichlet series. Then it would be of the form
		\begin{align*}
			\mathfrak{D}_{h}(s)
				& = \sum_{n\geq 1}\dfrac{h(n)}{n^{s}}
		\end{align*}
	The coefficients $h(n)$ of $\mathfrak{D}_{h}(s)$ is determined as follows.
		\begin{align*}
			h(n)
				& = \sum_{de=n}f(d)g(e)
		\end{align*}
	After a little observation, it seems quite obvious that this is indeed correct. In fact, this is what we call  \index{Dirichlet convolution}\textit{Dirichlet convolution} today.
		\begin{definition}[Dirichlet convolution]
			For two arithmetic functions $f$ and $g$, the \textit{Dirichlet product} or \textit{Dirichlet convolution} (or sometimes, \textit{Dirichlet product}) of $f$ and $g$ is defined as
				\begin{align*}
					f\ast g
						& = \sum_{d\mid n}f(d)g\left(\dfrac{n}{d}\right)
				\end{align*}
		\end{definition}
	
		\begin{theorem}
			Let $f$ and $g$ be multiplicative arithmetic functions. Then $f\ast g$ is also multiplicative.
		\end{theorem}
	
		\begin{proof}
			
		\end{proof}
	
		\begin{theorem}[Associativity of Dirichlet convolution]\label{thm:convass}
			Dirichlet convolution is associative. That is, if $f,g$ and $h$ are arithmetic functions, then
				\begin{align*}
					(f\ast g)\ast h
						& = f\ast (g\ast h)
				\end{align*}
		\end{theorem}
	
		\begin{proof}
			
		\end{proof}
	 An interesting function associated with Dirichlet convolution and summatory functions is the \index{M\"{o}bius function}\textit{M\"{o}bius function} $\mu$, defined in \textcite{mobius_1832}.
		\begin{align*}
			\mu(n)
				& =
					\begin{cases}
						0& \mbox{ if }p^{2}\mid n\mbox{ for some prime }p\\
						(-1)^{\omega(n)}
					\end{cases}
		\end{align*}
	where $\omega(n)$ is the number of distinct prime divisors of $n$. On the other hand, $\Omega(n)$ is the total number of prime divisors of $n$. So, $\omega(12)=2$ whereas $\Omega(12)=3$.
		\begin{theorem}[M\"{o}bius inversion]\label{thm:mobinv}
			Let $f$ be an arithmetic function and $F$ be the summatory function
				\begin{align*}
					F(n)
					& = \sum_{d\mid n}f(d)
				\end{align*}
			Then
				\begin{align*}
					f(n)
						& = \sum_{d\mid n}\mu(d)F\left(\dfrac{n}{d}\right)
				\end{align*}
		\end{theorem}
		
		\begin{proof}
			
		\end{proof}
	Following \textcite[Page $4$, Theorem $1.2.3$]{cojocaru_murty_2006}, let us define \index{Dual convolution}\textit{dual convolution}.
		\begin{definition}[Dual Convolution]
			Let $f$ and $g$ be arithmetic functions. Then the dual convolution of $f$ and $g$ is the arithmetic function $h$ defined as
				\begin{align*}
					h(n)
						& = \sum_{\substack{n\mid d\\d\in\mathbb{D}}}f(d)g\left(\dfrac{d}{n}\right)
				\end{align*}
			where $\mathbb{D}$ is a set of positive integers such that if $d\mid n$ and $n\in\mathbb{D}$, then $d\in\mathbb{D}$.
		\end{definition}
	Let us call such a set $\mathbb{D}$ \textit{divisor closed}.
		\begin{theorem}[Dual M\"{o}bius Inversion]
			Let $f$ be an arithmetic function and $F$ be the summatory function
				\begin{align*}
					F(n)
						& = \sum_{\substack{n\mid d\\d\in\mathbb{D}}}f(d)
				\end{align*}
			Then
				\begin{align*}
					f(n)
						& = \sum_{\substack{n\mid d\\d\in\mathbb{D}}}\mu\left(\dfrac{d}{n}\right)f(d)
				\end{align*}
		\end{theorem}
	
		\begin{proof}
			
		\end{proof}
\end{document}