\documentclass[elemeannt.tex]{subfile}

\begin{document}
	\nomenclature[a]{$\gcd(a,b)$}{Greatest common divisor of $a$ and $b$.}
	\nomenclature[b]{$\lcm(a,b)$}{Least common multiple of $a$ and $b$.}
	\nomenclature[c]{$\varphi(n)$}{Euler's totient function of $n$, $\varphi(n)$ is the number of positive integers not exceeding $n$ which are relatively prime to $n$.}
	\nomenclature[ca]{$J_{k}(n)$}{Jordan function of $n$, the number of tuples $(a_{1},\ldots,a_{k})$ such that $\gcd(a_{1},\ldots,a_{k},n)=1$ and $1\leq a_{1},\ldots,a_{k}\leq n$.}
    \nomenclature[d]{$\tau_{k}(n)$}{Generalized number of divisors of $n$, $\tau_{k}(n)=\sum_{d_{1}\cdots d_{k}=n}1$. For $k=1$, $\tau_{1}(n)=\tau(n)$, number of divisors of $n$.}
    \nomenclature[e]{$\sigma_{k}(n)$}{Generalized sum of divisors of $n$, $\sigma_{k}(n)=\sum_{d\mid n}d^{k}$. For $k=1$, $\sigma_{1}(n)=\sigma(n)$, sum of divisors of $n$.}
    \nomenclature[f]{$\omega(n)$}{Number of distinct prime divisors of $n$.}
    \nomenclature[g]{$\Omega(n)$}{Number of total prime divisors of $n$.}
    \nomenclature[ga]{$\floor{x}$}{Floor of $x$, greatest integer not exceeding $x$.}
    %\nomenclature[gaa]{$\ceiling{x}$}{Ceiling of $x$, smallest integer not less than $x$.}
    \nomenclature[gaaa]{$I(n)$}{Identity function, $I(n)=\floor{\frac{1}{n}}$.}
    \nomenclature[h]{$\mu(n)$}{M\"{o}bius function of $n$, $\mu(n)=(-1)^{\omega(n)}$ if $n$ is square-free, otherwise $\mu(n)=0$.}
    \nomenclature[ha]{$\lambda(n)$}{Liouville function of $n$, $\lambda(n)=(-1)^{\Omega(n)}$.}
    \nomenclature[hb]{$\Lambda(n)$}{Von Mangoldt Function of $n$. $\Lambda(n)=\log{p}$ if $n=p^{e}$ for some positive integer $e$, otherwise $\Lambda(n)=0$.}
    \nomenclature[i]{$\vartheta(x)$}{Tchebycheff function of the first kind.}
    \nomenclature[j]{$\psi(x)$}{Tchebycheff function of the second kind.}
    \nomenclature[k]{$\zeta(s)$}{Zeta function of the complex number $s$.}
    \nomenclature[ka]{$H(x)$}{Harmonic sum for $x$, $H(x)=\sum_{n\leq x}\frac{1}{x}$.}
    \nomenclature[l]{$\alpha\ast\beta$}{Dirichlet convolution of two arithmetic functions $\alpha$ and $\beta$.}
    \nomenclature[m]{$\alpha\circ\beta$}{General convolution of two arithmetic functions $\alpha$ and $\beta$.}
    \nomenclature[ma]{$\alpha\bullet\beta$}{Generalized convolution of two arithmetic functions $\alpha$ and $\beta$.}
    \nomenclature[n]{$\gamma$}{Euler-Mascheroni constant.}
\end{document}