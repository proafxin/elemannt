\documentclass[elemannt.tex]{subfile}

\begin{document}
	\chapter{Appendix}
		\begin{theorem}[Cauchy's convergence criteria]\label{thm:cauchyconvergence}
			Let $(a_{n\geq})$ be a sequence of complex numbers. Then $(a_{n\geq1})$ converges if and only if for any positive real number $\epsilon$, there exists a positive integer $N$ such that $s_{n}-s_{m}<\epsilon$ for all $n\geq m\geq N$ where
				\begin{align*}
					s_{k}
						& = \sum_{i=1}^{k}a_{i}
				\end{align*}
		\end{theorem}

		\begin{theorem}[Stirling's approximation formula]\label{thm:stirling}
			For positive integer $n$,
				\begin{align*}
					\sqrt{2\pi n}\left(\dfrac{n}{e}\right)^{n}e^{\dfrac{1}{12n+1}}
						& < n! < \sqrt{2\pi n}\left(\dfrac{n}{e}\right)^{n}e^{\dfrac{1}{12n}}
				\end{align*}
		\end{theorem}
	\textcite{stirling_1730} actually showed an exact series for $\log{n!}$ but usually a weaker statement like this suffices. We can also write it as
		\begin{align*}
			\log{n!}
				& = n\log{n}-n+\bigo{\log{n}}
		\end{align*}
\end{document}